%-------------------------
% Resume in Latex
% Author : Arjun Bhamra
% Based off of: https://github.com/sb2nov/resume
% License : MIT
%------------------------

\documentclass[letterpaper,11pt]{article}

\usepackage{latexsym}
\usepackage[empty]{fullpage}
\usepackage{titlesec}
\usepackage{marvosym}
\usepackage[usenames,dvipsnames]{color}
\usepackage{verbatim}
\usepackage{enumitem}
\usepackage[hidelinks]{hyperref}
\usepackage{fancyhdr}
\usepackage[english]{babel}
\usepackage{tabularx}
\input{glyphtounicode}
\setlength{\footskip}{4.08003pt}


%----------FONT OPTIONS----------
% sans-serif
% \usepackage[sfdefault]{FiraSans}
% \usepackage[sfdefault]{roboto}
% \usepackage[sfdefault]{noto-sans}
% \usepackage[default]{sourcesanspro}

% serif
% \usepackage{CormorantGaramond}
% \usepackage{charter}


\pagestyle{fancy}
\fancyhf{} % clear all header and footer fields
\fancyfoot{}
\renewcommand{\headrulewidth}{0pt}
\renewcommand{\footrulewidth}{0pt}

% Adjust margins
\addtolength{\oddsidemargin}{-0.5in}
\addtolength{\evensidemargin}{-0.5in}
\addtolength{\textwidth}{1in}
\addtolength{\topmargin}{-.5in}
\addtolength{\textheight}{1.0in}

\urlstyle{same}

\raggedbottom
\raggedright
\setlength{\tabcolsep}{0in}

% Sections formatting
\titleformat{\section}{
  \vspace{-11pt}\scshape\raggedright\large
}{}{0em}{}[\color{black}\titlerule \vspace{-3pt}]

% Ensure that generate pdf is machine readable/ATS parsable
\pdfgentounicode=1

%-------------------------
% Custom commands
\newcommand{\resumeItem}[1]{
  \item\small{
    {#1 \vspace{-1pt}}
  }
}

\newcommand{\resumeSubheading}[4]{
  \vspace{-2pt}\item
    \begin{tabular*}{0.97\textwidth}[t]{l@{\extracolsep{\fill}}r}
      \textbf{#1} & #2 \\
      \textit{\small#3} & \textit{\small #4} \\
    \end{tabular*}\vspace{-7pt}
}

\newcommand{\resumeSubSubheading}[2]{
    \item
    \begin{tabular*}{0.97\textwidth}{l@{\extracolsep{\fill}}r}
      \textit{\small#1} & \textit{\small #2} \\
    \end{tabular*}\vspace{-7pt}
}

\newcommand{\resumeProjectHeading}[2]{
    \item
    \begin{tabular*}{0.97\textwidth}{l@{\extracolsep{\fill}}r}
      \small#1 & #2 \\
    \end{tabular*}\vspace{-7pt}
}

\newcommand{\resumeSubItem}[1]{\resumeItem{#1}\vspace{-4pt}}

\renewcommand\labelitemii{$\vcenter{\hbox{\tiny$\bullet$}}$}

\newcommand{\resumeSubHeadingListStart}{\begin{itemize}[leftmargin=0.15in, label={}]}
\newcommand{\resumeSubHeadingListEnd}{\end{itemize}}
\newcommand{\resumeItemListStart}{\begin{itemize}}
\newcommand{\resumeItemListEnd}{\end{itemize}\vspace{-5pt}}

%-------------------------------------------
%%%%%%  RESUME STARTS HERE  %%%%%%%%%%%%%%%%%%%%%%%%%%%%


\begin{document}

%----------HEADING----------
% \begin{tabular*}{\textwidth}{l@{\extracolsep{\fill}}r}
%   \textbf{\href{http://sourabhbajaj.com/}{\Large Sourabh Bajaj}} & Email : \href{mailto:sourabh@sourabhbajaj.com}{sourabh@sourabhbajaj.com}\\
%   \href{http://sourabhbajaj.com/}{http://www.sourabhbajaj.com} & Mobile : +1-123-456-7890 \\
% \end{tabular*}

\begin{center}
  \textbf{\Huge {Arjun Bhamra}} \\ \vspace{1pt}
    \small U.S. Citizen $|$ 
    % abhamra@gatech.edu $|$ 
    \href{mailto:abhamra3@gatech.edu}{\underline{abhamra@gatech.edu}} 
    $|$ 
    % linkedin.com/in/arjun-bhamra $|$
    \href{https://linkedin.com/in/arjun-bhamra}{\underline{linkedin.com/in/arjun-bhamra}} $|$
   \href{https://github.com/abhamra}{\underline{github.com/abhamra}} $ | $
    \href{https://abhamra.com}{\underline{abhamra.com}}
    % \href{https://github.com/abhamra}{\underline{github.com/abhamra}}
\end{center}


%-----------EDUCATION-----------
\section{Education}
  \resumeSubHeadingListStart
    \resumeSubheading
      {Georgia Institute of Technology}{Atlanta, GA}
      {Bachelor of Science in Computer Science, GPA: 3.85}{ May 2026}

  \resumeSubHeadingListEnd
  \hspace{6pt} \textit{Relevant Coursework}: Design of Algorithms, Operating Systems, Graph Theory, Compilers (IP),
  \\ \hspace{6pt} Advanced Computer Organization (IP), Processor Design (IP)
%-----------EXPERIENCE-----------
\section{Experience}
  \resumeSubHeadingListStart

    \resumeSubheading
      {IBM Research}{May 2024 -- August 2024}
      {Software Engineering Intern}{Yorktown Heights, NY}
      \resumeItemListStart
        \resumeItem{Developed key components for IBM's Rust-based hardware-level production Quantum Assembly (QASM) quantum compiler that improved speed by 7x and memory usage by 3.5x}
        \resumeItem{Spearheaded the development of a 85x faster data parsing and caching API for quantum hardware configuration information and waveform data through binary (de)serialization}
        % \resumeItem{Created a 100x faster interface for storing and accessing cached configuration information through binary (de)serialization}
        \resumeItem{Implemented the compiler's intermediate representations (IRs) at the pulse calibration level}
        \resumeItem{Added additional support for QASM parsing, enabling extensibility and more language coverage}
      \resumeItemListEnd

    \resumeSubheading
      {Quantinuum}{May 2023 -- August 2023}
      {Software Engineering Intern}{Broomfield, CO}
      \resumeItemListStart
        \resumeItem{Created a Pythonic Domain Specific Language (DSL) to aid internal theory team in parsing code to Quantum Intermediate Representation (QIR) with a 30\% improvement in workflow efficiency}
        \resumeItem{Leveraged the Python Abstract Syntax Tree (AST) to build a high level compiler from custom language to LLVM IR that enables conditional branching on quantum measurements and dynamic circuits}
        \resumeItem{Added constant folding, constant propagation, method inlining and loop unrolling optimizations which reduced output QIR by 15\% for faster compilation}
      \resumeItemListEnd
      
% -----------Multiple Positions Heading-----------
    % \resumeSubSubheading
    % {Software Engineer I}{Oct 2014 - Sep 2016}
    % \resumeItemListStart
    %     \resumeItem{Apache Beam}
    %      {Apache Beam is a unified model for defining both batch and streaming data-parallel processing pipelines}
    % \resumeItemListEnd
    % \resumeSubHeadingListEnd
% -------------------------------------------

    % \resumeSubheading
    %   {STEMchats}{Sep. 2020 -- Jan 2022}
    %   {Software Engineering Intern}{Remote}
    %   \resumeItemListStart
    %     % \resumeItem{Built a website using Javascript, HTML, CSS, Node.js, Firebase, and Bootstrap to host tech newsletters}
    %     \resumeItem{Hosted weekly tech newsletters by building a website with Javascript, Node.js, Firebase, and Bootstrap}
    %     \resumeItem{Spearheaded the design of the main page, database, and article display systems for the newsletter search pages}
    %     \resumeItem{Delivered a 470\% boost in readership over the first 5 months of publishing the website compared to mailing list}
    % \resumeItemListEnd

    % \resumeSubheading
    %   {Lead Software Engineer}{August 2019 -- May 2022}
    %   {First Robotics Team 2234}{Newtown Square, PA}
    %   \resumeItemListStart
    %     \resumeItem{Developed a variety of complex drivetrain code (Swerve Drive, Tank, H Drive) in Java}
    %     \resumeItem{Utilized GRIP and regression calculations for precise autonomous ball launchers with an 80+\% accuracy rate}
    %     % \resumeItem{Mentored junior programming team and taught them Java, WPILib, GRIP, and Git best practices}
    %   \resumeItemListEnd

  \resumeSubHeadingListEnd

%-----------RESEARCH-----------
\section{Research}
  \resumeSubHeadingListStart
      \resumeSubheading
          {Quantum Programming Languages Researcher}{July 2023 -- Present}
          {Georgia Tech CRNCH-TINKER Lab - Dr. Tom Conte}{Atlanta, GA}
          \resumeItemListStart
            \resumeItem{Aided in the design and implementation of Qwerty, a quantum programming language that moves away from gate-based quantum computing and towards higher abstractions}
            \resumeItem{Engineered custom ASTs and optimization passes for the Qwerty compiler using the Python AST, C++, and custom MLIR dialects and rewrite passes}
            \resumeItem{Placed 2nd at the CGO Student Research Competition for \textbf{“Type Inference for Qwerty”} by implementing Hindley Milner type inference via a Rust-based AST}
            \resumeItem{Austin J. Adams, Sharjeel Khan, \textbf{Arjun Bhamra}, Ryan Abusaada, Anthony M. Cabrera, Cameron Hoechst, Jeffrey S. Young, and Thomas M. Conte. \textbf{“ASDF: A Compiler for Qwerty, a Basis-Oriented Quantum Programming Language.”} \textit{2025 IEEE/ACM International Symposium on Code Generation and Optimization (CGO ‘25).}}
            
          \resumeItemListEnd

    % \resumeSubheading
    %   {Quantum Algorithms Researcher}{Aug 2022 -- July 2023}
    %   {Georgia Tech Computational Physics Lab - Dr. Spencer Bryngelson}{Atlanta, GA}
    %   \resumeItemListStart
    %     \resumeItem{Created two variational quantum algorithms (VQAs) for solving PDEs with Qiskit, etc.}
    %     \resumeItem{Expanded upon Variational Quantum Linear Solver (VQLS) and VQE for fluid dynamics applications}
        
    %   \resumeItemListEnd

    % \resumeSubheading
    %   {Quantum Machine Learning Researcher}{August 2020 -- May 2022}
    %   {Aspiring Scholars Directed Research Program (ASDRP)}{Newtown Square, PA}
    %   \resumeItemListStart
    %     \resumeItem{Utilized Pennylane and PyTorch to design a novel Hybrid Quantum-Classical Generative Adversarial Network (QGAN) that outputs new chemically stable molecules}
    %     \resumeItem{Published in the \href{https://emerginginvestigators.org/articles/hybrid-quantum-classical-generative-adversarial-network-for-synthesizing-chemically-feasible-molecules/pdf}{\underline{Journal of Emerging Investigators}}}
        
    %   \resumeItemListEnd

  \resumeSubHeadingListEnd

%-----------LEADERSHIP---------
% \section{Leadership}
%     \resumeSubHeadingListStart
%       \resumeProjectHeading
%           {\textbf{Georgia Tech Esports} $|$ \emph{Co-President}}{}
%           \resumeItemListStart
%             \resumeItem{Managed casual and competitive scenes for 15+ games under the GT professional banner}
%             \resumeItem{Aided in implementing a long lasting, effective organizational structure for future generations}
%           \resumeItemListEnd
%     \resumeSubHeadingListEnd

%-----------PROJECTS-----------
\section{Projects}
    \resumeSubHeadingListStart
          % \resumeProjectHeading{\textbf{Quill: Quantum Programming from the Middle Ages} $|$ \emph{Rust}}{}
          % \resumeItemListStart
          %   \resumeItem{Created a quantum programming language using Shakespearean syntax, with a streamlined parsing and AST generation process through the Pest crate }
          %   \resumeItem{Laid out a plan to implement peephole optimization and code generation through QIR}
          % \resumeItemListEnd
     \resumeProjectHeading{\textbf{Open Source Contributor to Quantum repositories} $|$ \emph{Rust, Python, unittest, Git}}{}
          \resumeItemListStart
            \resumeItem{Added necessary quality of life changes for the IBM Rustworkx repository, a Rust-based Python graph library used in Qiskit}
            \resumeItem{Added major feature support for the IBM OpenQASM 3 Parser, a performant parser based on rust-analyzer}
            \resumeItem{Provided bug fixes and feature improvements for Xanadu's JIT quantum compiler, Catalyst}
          \resumeItemListEnd
          \resumeProjectHeading{\textbf{rs\_micrograd: Mini Automatic Differentiation Engine} $|$ \emph{Rust}}{}
          \resumeItemListStart
            \resumeItem{Designed an auto-diff engine with Rust based on Andrej Karpathy's Python micrograd engine}
            \resumeItem{Implemented expression graph-based gradient calculations for efficient neural network backpropagation}
          \resumeItemListEnd
    \resumeSubHeadingListEnd

%-----------PROGRAMMING SKILLS-----------
\section{Technical Skills}
 \begin{itemize}[leftmargin=0.15in, label={}]
    \small{\item{
     \textbf{Languages}{: Rust, Python, Java, C, C++, LLVM, MLIR, JavaScript, LateX} \\
     % \textbf{Frameworks}{: React, Node.js, Bootstrap} \\
     \textbf{Developer Tools}{: Git, Gitlab, Google Cloud Platform, BitBucket, Atlassian Tools, Bazel} \\
     \textbf{Libraries}{: pandas, NumPy, Matplotlib, unittest, Qiskit, Pennylane, PyTorch, Scikit-learn, Python AST, functools}
    }}
 \end{itemize}
 
 %-----------REFERENCES----------\ 
% \section{References}
%  \begin{itemize}[leftmargin=0.15in, label={}]
%     \small{\item{\textbf{Dr. Spencer Bryngelson - Georgia Tech Computational Physics Lab} \par
%     shb@gatech.edu - (734) 674-5501}}
%     \small{\item{\textbf{Dr. Bryan Gard - Georgia Tech Research Institute} \par
%     Bryan.Gard@gtri.gatech.edu - (225) 324-3914}}
%     \small{\item{\textbf{Brady Werkheiser - Alchemy} \par
%     brady@alchemy.com - (215) 264-3202}}
%  \end{itemize}


%-------------------------------------------
\end{document}
