%-------------------------
% Resume in Latex
% Author : Arjun Bhamra
% Based off of: https://github.com/sb2nov/resume
% License : MIT
%------------------------

\documentclass[letterpaper,11pt]{article}

\usepackage{latexsym}
\usepackage[empty]{fullpage}
\usepackage{titlesec}
\usepackage{marvosym}
\usepackage[usenames,dvipsnames]{color}
\usepackage{verbatim}
\usepackage{enumitem}
\usepackage[hidelinks]{hyperref}
\usepackage{fancyhdr}
\usepackage[english]{babel}
\usepackage{tabularx}
\input{glyphtounicode}


\pagestyle{fancy}
\fancyhf{} % clear all header and footer fields
\fancyfoot{}
\renewcommand{\headrulewidth}{0pt}
\renewcommand{\footrulewidth}{0pt}

% Adjust margins
\addtolength{\oddsidemargin}{-0.5in}
\addtolength{\evensidemargin}{-0.5in}
\addtolength{\textwidth}{1in}
\addtolength{\topmargin}{-.5in}
\addtolength{\textheight}{1.0in}

\urlstyle{same}

\raggedbottom
\raggedright
\setlength{\tabcolsep}{0in}

% Sections formatting
\titleformat{\section}{
  \vspace{-11pt}\scshape\raggedright\large
}{}{0em}{}[\color{black}\titlerule \vspace{-3pt}]

% Ensure that generate pdf is machine readable/ATS parsable
\pdfgentounicode=1

%-------------------------
% Custom commands
\newcommand{\resumeItem}[1]{
  \item\small{
    {#1 \vspace{-1pt}}
  }
}

\newcommand{\resumeSubheading}[4]{
  \vspace{-2pt}\item
    \begin{tabular*}{0.97\textwidth}[t]{l@{\extracolsep{\fill}}r}
      \textbf{#1} & #2 \\
      \textit{\small#3} & \textit{\small #4} \\
    \end{tabular*}\vspace{-7pt}
}

\newcommand{\resumeSubSubheading}[2]{
    \item
    \begin{tabular*}{0.97\textwidth}{l@{\extracolsep{\fill}}r}
      \textit{\small#1} & \textit{\small #2} \\
    \end{tabular*}\vspace{-7pt}
}

\newcommand{\resumeProjectHeading}[2]{
    \item
    \begin{tabular*}{0.97\textwidth}{l@{\extracolsep{\fill}}r}
      \small#1 & #2 \\
    \end{tabular*}\vspace{-7pt}
}

\newcommand{\resumeSubItem}[1]{\resumeItem{#1}\vspace{-4pt}}

\renewcommand\labelitemii{$\vcenter{\hbox{\tiny$\bullet$}}$}

\newcommand{\resumeSubHeadingListStart}{\begin{itemize}[leftmargin=0.15in, label={}]}
\newcommand{\resumeSubHeadingListEnd}{\end{itemize}}
\newcommand{\resumeItemListStart}{\begin{itemize}}
\newcommand{\resumeItemListEnd}{\end{itemize}\vspace{-5pt}}

%-------------------------------------------
%%%%%%  RESUME STARTS HERE  %%%%%%%%%%%%%%%%%%%%%%%%%%%%

% to fix the compilation warnings
\setlength{\footskip}{5pt}
\begin{document}

%----------HEADING----------

\begin{center}
    \textbf{\Huge Arjun Bhamra} \\ \vspace{1pt}
    \small U.S. Citizen $|$ 
    % abhamra@gatech.edu $|$ 
    \href{mailto:abhamra3@gatech.edu}{\underline{abhamra@gatech.edu}} 
    $|$ 
    \href{https://linkedin.com/in/arjun-bhamra}{\underline{linkedin.com/in/arjun-bhamra}} $|$
   \href{https://github.com/abhamra}{\underline{github.com/abhamra}} $ | $
    \href{https://abhamra.com}{\underline{abhamra.com}}
    % \href{https://github.com/abhamra}{\underline{github.com/abhamra}}
\end{center}

\section{Research Interests}
  \resumeSubHeadingListStart
  \item Compilers, programming languages, and quantum computation

  \resumeSubHeadingListEnd

%-----------EDUCATION-----------
\section{Education}
  \resumeSubHeadingListStart
    \resumeSubheading
      {Georgia Institute of Technology}{Atlanta, GA}
      {Bachelor of Science in Computer Science, GPA: 3.87}{ May 2026}
  \resumeSubHeadingListEnd
  \hspace{6pt} \textit{Threads}: Theory, Systems \& Architecture

%-----------EXPERIENCE-----------
\section{Experience}
  \resumeSubHeadingListStart
    \resumeSubheading
      {IBM Research}{May 2025 -- August 2025}
      {Research Intern}{Yorktown Heights, NY}
      \resumeItemListStart
        \resumeItem{Designed a new qubit-efficient quantum algorithm for solving the Traveling Salesman Problem}
        \resumeItem{Leveraged permutation encoding and Pauli Correlation Encoding, along with a K-Means path slicing strategy to reduce qubit overhead by 10x, paving the way for solving utility scale problems on quantum hardware}
        \resumeItem{Paper in progress, in collaboration with IBM partners}
        \resumeItem{Operated on TSPLIB instances orders of magnitude larger than existing hybrid-quantum methods}
      \resumeItemListEnd

    \resumeSubheading
      {IBM Research}{May 2024 -- August 2024}
      {Software Engineering Intern}{Yorktown Heights, NY}
      \resumeItemListStart
        \resumeItem{Developed key components for IBM's Rust-based hardware-level production Quantum Assembly (QASM) quantum compiler that improved speed by 7x and memory usage by 3.5x}
        \resumeItem{Spearheaded the development of a 85x faster data parsing and caching API for quantum hardware configuration information and waveform data through binary (de)serialization}
        \resumeItem{Implemented the compiler's intermediate representations (IRs) at the pulse calibration level}
        \resumeItem{Added additional support for QASM parsing, enabling extensibility and more language coverage}
      \resumeItemListEnd

    \resumeSubheading
      {Quantinuum}{May 2023 -- August 2023}
      {Software Engineering Intern}{Broomfield, CO}
      \resumeItemListStart
        \resumeItem{Created a Pythonic Domain Specific Language (DSL) to aid internal theory team in parsing code to Quantum Intermediate Representation (QIR) with a 30\% improvement in workflow efficiency}
        \resumeItem{Leveraged the Python Abstract Syntax Tree (AST) to build a high level compiler from custom language to LLVM IR that enables conditional branching on quantum measurements and dynamic circuits}
        \resumeItem{Added constant folding, constant propagation, method inlining and loop unrolling optimizations which reduced output QIR by 15\% for faster compilation}
      \resumeItemListEnd
    

  \resumeSubHeadingListEnd

%-----------RESEARCH-----------
\section{Research Experience}
  \resumeSubHeadingListStart
      \resumeSubheading
          {Quantum Programming Languages Researcher}{July 2023 -- Present}
          {Georgia Tech CRNCH-TINKER Lab - Dr. Tom Conte}{Atlanta, GA}
          \resumeItemListStart
            \resumeItem{Aided in the design and implementation of Qwerty, a quantum programming language that moves away from gate-based quantum computing and towards higher abstractions}
            \resumeItem{Engineered custom ASTs and optimization passes for the Qwerty compiler using the Python AST, C++, and custom MLIR dialects and rewrite passes}
            \resumeItem{Placed 2nd at the CGO Student Research Competition for \textbf{"Type Inference for Qwerty"} by implementing Hindley Milner type inference via a Rust-based AST}
            \resumeItem{Designed and implemented efficient circuit synthesis for Qwerty's "basis generators", enabling cleaner writing of generic Quantum Fourier Transform (QFT) circuits with recursive structure}
            
          \resumeItemListEnd

    \resumeSubheading
      {Quantum Algorithms Researcher}{Aug 2022 -- July 2023}
      {Georgia Tech Computational Physics Lab - Dr. Spencer Bryngelson}{Atlanta, GA}
      \resumeItemListStart
        \resumeItem{Created two variational quantum algorithms (VQAs) for solving PDEs with Qiskit, etc.}
        \resumeItem{Expanded upon Variational Quantum Linear Solver (VQLS) and VQE for fluid dynamics applications}
        
      \resumeItemListEnd

  \resumeSubHeadingListEnd

\section{Publications}
    \resumeSubHeadingListStart
      \resumeItem{Austin J. Adams, Sharjeel Khan, \textbf{Arjun S. Bhamra}, Ryan R. Abusaada, Jeffrey S. Young, and Thomas M. Conte. \textbf{“Qwerty: A Basis-Oriented Quantum Programming Language”} \textit{IEEE International Conference on Quantum Computing and Engineering (QCE25).}}
      \resumeItem{Austin J. Adams, Sharjeel Khan, \textbf{Arjun Bhamra}, Ryan Abusaada, Anthony M. Cabrera, Cameron Hoechst, Jeffrey S. Young, and Thomas M. Conte. \textbf{“ASDF: A Compiler for Qwerty, a Basis-Oriented Quantum Programming Language”} \textit{2025 IEEE/ACM International Symposium on Code Generation and Optimization (CGO ‘25).}}
    \resumeSubHeadingListEnd

%-----------PROJECTS-----------
\section{Projects}
    \resumeSubHeadingListStart

     \resumeProjectHeading{\textbf{Open Source Contributor to Quantum repositories} $|$ \emph{Rust, Python, unittest, Git}}{}
          \resumeItemListStart
            \resumeItem{Added necessary quality of life changes for the IBM Rustworkx repository, a Rust-based Python graph library}
            \resumeItem{Added major feature support for the IBM OpenQASM 3 Parser, a performant parser based on rust-analyzer}
            \resumeItem{Provided bug fixes and feature improvements for Xanadu's JIT quantum compiler, Catalyst}
          \resumeItemListEnd
          \resumeProjectHeading{\textbf{rs\_micrograd: Mini Automatic Differentiation Engine} $|$ \emph{Rust}}{}
          \resumeItemListStart
            \resumeItem{Designed an auto-diff engine with Rust based on Andrej Karpathy's Python micrograd engine}
            \resumeItem{Implemented expression graph-based gradient calculations for efficient neural network backpropagation}
          \resumeItemListEnd
    \resumeSubHeadingListEnd

% -----------LEADERSHIP---------
\section{Leadership}
    \resumeSubHeadingListStart
      \resumeProjectHeading
          {\textbf{Dependently Typed: Programming Languages and Compilers Club} $|$ \emph{Co-President} $|$ 2024 - Present}{}
          \resumeItemListStart
            \resumeItem{Organized/coordinated talks from industry experts, professors, and students}
            \resumeItem{Gave talks on MLIR, Quantum Programming Languages, and Program Synthesis}
            \resumeItem{Mentored students in programming languages and compilers as a research field, enabling future success and open communication}
          \resumeItemListEnd
      \resumeProjectHeading
          {\textbf{Georgia Tech Esports} $|$ \emph{Co-President} $|$ 2023 - 2024}{}
          \resumeItemListStart
            \resumeItem{Managed casual and competitive scenes for 15+ games under the GT professional banner}
            \resumeItem{Aided in implementing a long lasting, effective organizational structure for future generations}
          \resumeItemListEnd
    \resumeSubHeadingListEnd

%-----------PROGRAMMING SKILLS-----------
\section{Technical Skills}
 \begin{itemize}[leftmargin=0.15in, label={}]
    \small{\item{
     \textbf{Languages}{: Rust, Python, Java, C, C++, LLVM, MLIR, JavaScript, LateX} \\
     \textbf{Libraries}{: pandas, NumPy, Matplotlib, unittest, Qiskit, Pennylane, PyTorch, Scikit-learn, Python AST, functools}
    }}
 \end{itemize}
\end{document}
end{document}
